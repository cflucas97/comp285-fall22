 \documentclass [12pt]{article} 

\usepackage {amsmath}
\usepackage {amsthm}
\usepackage {amssymb}
\usepackage {graphicx} 
\usepackage {float}
\usepackage {multirow}
\usepackage {xcolor}
\usepackage {algorithmic}
\usepackage [ruled,vlined,commentsnumbered,titlenotnumbered]{algorithm2e} \usepackage {array} 
\usepackage {booktabs} 
\usepackage {url} 
\usepackage {parskip} 
\usepackage [margin=1in]{geometry} 
\usepackage [T1]{fontenc} 
\usepackage {cmbright} 
\usepackage [many]{tcolorbox} 
\usepackage [colorlinks = true,
            linkcolor = blue,
            urlcolor  = blue,
            citecolor = blue,
            anchorcolor = blue]{hyperref} 
\usepackage {enumitem} 
\usepackage {xparse} 
\usepackage {verbatim}
\usepackage{listings}
\usepackage{xcolor}
\lstset { %
    language=C++,
    backgroundcolor=\color{black!5}, % set backgroundcolor
    basicstyle=\footnotesize,% basic font setting
}
\newtheorem{theorem}{Theorem}
\newtheorem{remark}{Remark}
\newtheorem{lemma}[theorem]{Lemma}
\theoremstyle{definition}
\newtheorem{definition}{Definition}[section]
\newtheorem{claim}{Claim}




\DeclareTColorBox {Solution}{}{breakable, title={Solution}} \DeclareTColorBox {Solution*}{}{breakable, title={Solution (provided)}} \DeclareTColorBox {Instruction}{}{boxrule=0pt, boxsep=0pt, left=0.5em, right=0.5em, top=0.5em, bottom=0.5em, arc=0pt, toprule=1pt, bottomrule=1pt} \DeclareDocumentCommand {\Expecting }{+m}{\textbf {[We are expecting:} #1\textbf {]}} \DeclareDocumentCommand {\Points }{m}{\textbf {(#1 pt.)}} 

\begin {document} 

\vspace {1em} 
\begin {Instruction} 
Adapted From Virginia Williams' lecture notes.
\end {Instruction}  

{\LARGE \textbf {COMP 285 (NC A\&T, Spr `22)}\hfill \textbf {Lecture 22} } 

\begin{centering}
\section*{Single-Source Shortest Path in Weighted Graphs}
\end{centering}


\section{Dijkstra's Algorithm}

Now we will solve the single source shortest paths problem in graphs with nonnengative
weights using Dijkstra's algorithm. The key idea, that Dijkstra will maintain as an invariant,
is that $\forall t in V$, the algorithm computes an estimate $d[t]$ of the distance of $t$ from the source such that:

\begin{enumerate}
    \item At any point in time, $d[t] \geq d(s, t)$, and
    \item when t is finished, $d[t] = d(s, t)$.
\end{enumerate}


\begin{algorithm}
\caption{Dijkstra($G= (V,E), S$)}
\label{alg:1}
\begin{algorithmic}
\STATE $\forall t \in V, d[t] \gets \infty$ \texttt{// set initial distance estimates}
\STATE $d[s] \gets 0$
\STATE $F \gets \{v \mid \forall v \in V\}$ \texttt{// F is the set of nodes that are yet to achieve final distances estimates}
\STATE $D \gets \emptyset$ \texttt{// D will be the set of nodes that have achieved final distance estimates}
\WHILE{$F \neq \emptyset$}
    \STATE $x \gets$ elements in $F$ with minimum distance estimate
    \FOR{$(x,y) \in E$}
        \STATE $d[y] \gets \min\{d[y], d[x] + w(x,y)\}$ \texttt{// "relax" the estimate of y}
        \STATE \texttt{// to maintain paths: if} $d[y]$ \texttt{changes, then } $\pi(y) \gets x$
    \ENDFOR
    \STATE $F \gets F \setminus \{x\}$
    \STATE $D \gets D \cup \{x\}$
\ENDWHILE
\end{algorithmic}
\end{algorithm}

\begin{claim}[For every $u$, at any point of time $d(u) \geq d(s, u)$.]
\vspace{1em}
A formal proof of this claim proceeds by induction. In particular, one shows that at any point in time, if $d[u] < \infty$, then $d[u]$ is the weight of some path from $s$ to $t$. Thus at any point $d[u]$ is at least the weight of the shortest path, and hence $d[u] \geq d(s, u)$. As a base case, we know that $d[s] = 0 = d(s, s)$ and all other distance estimates are $+\infty$, so we know that the claim holds initially. Now, when $d[u]$ is changed to $d[x] + w(x, u)$ then (by the induction hypothesis) there is a path from $s$ to $x$ of weight $d[x]$ and an edge $(x, u)$ of weight $w(x, u)$. This means there is a path from $s$ to $u$ of weight $d[u] = d[x] + w(x, u)$. This implies that $d[u]$ is at least the weight of the shortest path $= d(s, u)$, and the induction argument is complete
\end{claim}


\begin{claim}[When node $x$ is placed in $D$, $d(x) = d(s,x)$] 
\vspace{1em}

Notice that proving the above claim is sufficient to prove the correctness of the algorithm since $d[x]$ is never changed again after $x$ is added to $D$: the only way it could be changed is if for some node $y \in F$ , $d[y] + w(y, x) < d[x]$ but this can’t happen since $d[x] \leq d[y ]$ and $w(y, x) \geq 0$ (all edge weights are nonnegative). The assertion $d[x] \leq d[y]$ for all $y \in F$ stays true at all points after $x$ is inserted into D: assume for contradiction that at some point for some $y \in F$ we get $d[y ] < d[x]$ and let $y$ be the first such $y$ . $Before d[y ]$ was updated $d[y' ] \geq d[x]$ for all $y' \in F$ . But then when $d[y ]$ was changed, it was due to some neighbor $y'$ of $y$ in $F$ , but$ d[y' ] \geq d[x]$ and all weights are nonnegative, so we get a contradiction 

We prove this claim by induction on the order of placement of nodes into $D$. For the base case, $s$ is placed into D where $d[s] = d(s, s) = 0$, so initially, the claim holds. 

For the inductive step, we assume that for all nodes $y$ currently in $D$, $d[y ] = d(s, y )$. Let $x$ be the node that currently has the minimum distance estimate in $F$ (this is the node about to be moved from $F$ to $D$). We will show that $d[x] = d(s, x)$ and this will complete the induction. Let $p$ be a shortest path from $s$ to $x$. Suppose $z$ is the node on $p$ closest to $x$ for which $d[z] = d(s, z)$. We know $z$ exists since there is at least one such node, namely $s$, where $d[s] = d(s, s)$. By the choice of $z$, for every node $y$ on $p$ between $z$ (not inclusive) to $x$ (inclusive), $d[y ] > d(s, y )$. Consider the following options for $z$.

\begin{enumerate} 
    \item If $z = x$, then $d[x] = d(s, x)$ and we are done.
    \item Suppose $z \neq x$. Then there is a node $z'$ after $z$ on $p$. (Here it is possible that $z' = x$.) We know that $d[z] = d(s, z) \leq d(s, x) \leq d[x]$. The first $\leq$ inequality holds because subpaths of shortest paths are shortest paths as well, so that the prefix of $p$ from $s$ to $z$ has weight $d(s, z)$. In addition, the weights on edges are non-negative, so that the portion of $ p$ from $z$ to $x$ has a nonnegative weight, and so $d(s, z) \leq d(s, x)$. The subsequent $\leq $ holds by Claim 1. We know that if $d[z] = d[x]$ all of the previous inequalities are equalities and $d[x] = d(s, x)$ and the claim holds. 

    Finally, towards a contradiction, suppose $d[z] < d[x]$. By the choice of $x \in F$ we know $d[x]$ is the minimum distance estimate that was in $F$ . Thus, since $d[z] < d[x]$, we know $z \notin F$ and must be in $D$, the finished set. This means the edges out of $z$, and in particular ($z, z' )$, were already relaxed by our algorithm. But this means that $d[z ' ] \leq d(s, z) + w(z, z' ) = d(s, z' )$, because $z$ is on the shortest path from $s$ to $z '$ , and the distance estimate of $z '$ must be correct. However, this contradicts $z$ being the closest node on $p$ to $x$ meeting the criteria$ d[z] = d(s, z)$. Thus, our initial assumption that $d[z] < d[x]$ must be false and $d[x]$ must equal $d(s, x)$.
\end{enumerate}
\end{claim}


\subsection{Implementation of Dijkstra's Algorithm}


\end{document}