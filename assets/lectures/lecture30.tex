 \documentclass [12pt]{article} 

\usepackage {amsmath}
\usepackage {amsthm}
\usepackage {amssymb}
\usepackage {graphicx} 
\usepackage {float}
\usepackage {multirow}
\usepackage {xcolor}
\usepackage [ruled,vlined,commentsnumbered,titlenotnumbered]{algorithm2e} \usepackage {array} 
\usepackage {booktabs} 
\usepackage {url} 
\usepackage {parskip} 
\usepackage [margin=1in]{geometry} 
\usepackage [T1]{fontenc} 
\usepackage {cmbright} 
\usepackage [many]{tcolorbox} 
\usepackage [colorlinks = true,
            linkcolor = blue,
            urlcolor  = blue,
            citecolor = blue,
            anchorcolor = blue]{hyperref} 
\usepackage {enumitem} 
\usepackage {xparse} 
\usepackage {verbatim}
\usepackage{algpseudocode}
\usepackage{listings}
\usepackage{xcolor}
\lstset { %
    language=C++,
    backgroundcolor=\color{black!5}, % set backgroundcolor
    basicstyle=\footnotesize,% basic font setting
}
\newtheorem{theorem}{Theorem}
\newtheorem{remark}{Remark}
\newtheorem{lemma}[theorem]{Lemma}
\theoremstyle{definition}
\newtheorem{definition}{Definition}[section]
\newtheorem{claim}{Claim}
\newtheorem{proposition}{Proposition}






\DeclareTColorBox {Solution}{}{breakable, title={Solution}} \DeclareTColorBox {Solution*}{}{breakable, title={Solution (provided)}} \DeclareTColorBox {Instruction}{}{boxrule=0pt, boxsep=0pt, left=0.5em, right=0.5em, top=0.5em, bottom=0.5em, arc=0pt, toprule=1pt, bottomrule=1pt} \DeclareDocumentCommand {\Expecting }{+m}{\textbf {[We are expecting:} #1\textbf {]}} \DeclareDocumentCommand {\Points }{m}{\textbf {(#1 pt.)}} 

\begin {document} 

\vspace {1em} 
\begin {Instruction} 
Adapted From Virginia Williams' lecture notes.
\end {Instruction}  

{\LARGE \textbf {COMP 285 (NC A\&T, Spr `22)}\hfill \textbf {Lecture 28} } 

\begin{centering}
\section*{More Greedy: Huffman Coding and MST}
\end{centering}


\section{Activity Selection}

Last lecture, we introduced the activity selection problem and walkted through a few possible candidates for greedy algorithms. 

\begin{proposition}
For each $S_{i ,j}$, there is an optimal solution $A_{i ,j}$ containing $a_k \in S_{i ,j}$ of minimum finishing time $f_k$.
\end{proposition}
 

Note that if the proposition is true, when $f_k$ is minimum, then $A_{i ,k}$ is empty, as no activities can finish before $a_k$ ; thus, our optimal solution only depends on one other subproblem $A_{k ,j}$ (giving us a linear time algorithm). 


Here, we prove the proposition.

\begin{proof}

Let $a_k$ be the activity of minimum finishing time in $S_{i ,j}$. Let $A_{i ,j}$ be some maximum set of non-conflicting activities. Consider $A'_{i ,j} = A_{i ,j} \setminus {a_l} \cup {a_k}$ where $a_l$ is the activity of minimum finishing time in $A_{i ,j}$. It’s clear that $|A'_{i ,j}| = |A_{i ,j}|$. We need to show that $A'_{i ,j}$ does not have conflicting activities. We know $a_l \in A_{i ,j} \subset S_{i ,j}$. This implies $f_l \geq f_k$ , since $a_k$ has the minimum finishing time in $S_{i ,j}$. 

All $a_t \in A_{i ,j} \setminus {a_l}$ don’t conflict with $a_l $, which means that $s_t \geq f_l$ , which means that $s_t \geq f_k$ , so this means that no activity in $A_{i ,j} \setminus {a_l}$ can conflict with $a_k$ . Thus, $A'_{i ,j}$ is an optimal solution.
\end{proof}

Due to the above proposition, the expression for $A_{i ,j}$ from before simplifies to the following
expression in terms of $a_k \subseteq S_{i ,j}$, the activity with minimum finishing time $f_k$ .

\begin{align*}
|A_{i,j}| = 1 + |A_{k,j}| \\
A_{i,j} = A_{k,j} \cup \{a_k \}
\end{align*}

Algorithm Greedy-AS assumes that the activities are presorted in nondecreasing order of their finishing time, so that if $i < j$, $f_i \leq f_j$.

\begin{algorithm}
\caption{Greedy-AS(a)}
\label{alg:greed_as}
\begin{algorithmic}
\State $A \gets \{a_1\}$ \texttt{/* activity of min $f_i$}
\State $k \gets 1$
\State \For{$m = 2 \to n$} {
    \State \If{$s_m \geq f_k$} {
        \State \texttt{// $a_m$ starts after last activity in A}
        \State $A \gets A \cup \{a_m\}$
        \State $k \gets m$
    }
}
\State \Return $A$
\end{algorithmic}
\end{algorithm}


By the above claim, this algorithm will produce a legal, optimal solution via a greedy selection of activities. There may be multiple optimal solutions, but there always exists a solution that includes $a_k$ with the minimum finishing time. The algorithm does a single pass over the activities, and thus only requires $O(n)$ time – a dramatic improvement from the trivial dynamic programming solution. If the algorithm also needed to sort the activities by $f_i$ , then its runtime would be $O(n \log n)$ which is still better than the original dynamic programming solution.



\section{Greedy Algorithms}






























\end{document}
